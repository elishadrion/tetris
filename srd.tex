\documentclass[11pt,a4paper]{article}

\usepackage[T1]{fontenc}
\usepackage[utf8]{inputenc}
\usepackage[frenchb]{babel} % Global stuff set to french
\usepackage[margin=2.5cm]{geometry} % The margin of the page
%\usepackage{amsmath}  % to include math formulas
\usepackage{graphicx} % to include pictures
\usepackage[hidelinks]{hyperref} % To include hyperlinks in a PDF
\usepackage{fancyhdr} % to be able to make the page fancy looking
\usepackage{lastpage} % so latex knows what is the last page...
\usepackage{appendix} % To make appendixes
\usepackage{color} % For text colors
\usepackage{palatino} % Change font
%\usepackage{tabularx}
\usepackage{changepage}
\usepackage{subcaption}
\usepackage{enumitem}

%% Fancy layout
\pagestyle{fancy}
    \lhead{Projet d'année - Partie 1}
    \chead{}
    \rhead{Groupe 2}
    \lfoot{}
    \cfoot{}
    \rfoot{Page \thepage\ de \pageref{LastPage}}
\renewcommand{\headrulewidth}{0.4pt}
\renewcommand{\footrulewidth}{0.4pt}


%%% --- %%% --- DOCUMENT START --- %%% --- %%%
\begin{document}
    \begin{titlepage}

\topskip0pt
\begin{center}
    \vspace*{\fill}
        \hrule
        \vspace*{2pt}
        \hrule
        \vspace*{15pt}
        \textsc{\Huge{INFO-F106 : Projet d'année \\\vspace*{8pt} Rappport intermédiaire}}
        \vspace*{15pt}
        \hrule
        \vspace*{2pt}
        \hrule
  \vspace*{\fill}
\end{center}
\null
\vfill
  
\large{Mardi 16 décembre 2014} \hfill \large{Carlos Requena López - \emph{410031}}

\end{titlepage}
    \pagestyle{empty}
\tableofcontents
\newpage
 %%% Counting pages now %%%
\pagestyle{fancy}

\setcounter{page}{1}

\section{Introduction}
\label{sec:intro}



\subsection{But du projet}
\label{sec:but}

% Texte (d’une demi-page approximativement) décrivant les tenants et
% aboutissants du projet, et énumérant les différents types de
% personnes qui vont bénéficier de la réalisation du projet.

\subsection{Glossaire}
\label{sec:glo}

% Définition des termes, des acronymes et des abréviations utilisés
% dans le présent document.




\subsection{Historique du document}
\label{sec:hist}

% Tableau reprenant tous les changements effectués sur le
% document. Chaque ligne de celui-ci contiendra les champs suivants :
% numéro de version, auteur et date de la modification, description
% des changements. Le tableau sera trié de manière décroissante sur le
% numéro de version.

% Example de versioning:


    % 0 for alpha (status)
    % 1 for beta (status)
    % 2 for release candidate
    % 3 for (final) release


\begin{table}[h]
  \centering
  \begin{tabular}[ht]{|l|l|l|p{18em}|}
    \hline

    \textbf{Version}
    & \textbf{Auteur}
    & \textbf{Date modification}
    & \textbf{Description des changements}\\ \hline \hline
    NEXT &  &  &  \\ \hline
    NEXT &  &  &  \\ \hline
    v0.0 & Carlos Requena & 26/11/15 22:30
    & Mis en place. Diagrammes realisés en groupe le 25/11/15 ajoutés\\ \hline
  \end{tabular}
  \caption{Changements document}
  \label{tab:hist}
\end{table}

\section{Besoins de l'utilisateur}
\label{sec:besoins}

% Cette deuxième section reprend les services que le système doit
% fournir aux utilisateurs, ainsi que les contraintes de
% fonctionnement du système. Elle doit être complète et consistante,
% et il faut qu’elle soit rédigée dans un langage spécialement
% compréhensible pour le client, c’est-à-dire en évitant entre autres
% tout terme ou détail techniques.

\subsection{Exigences fonctionnelles}
\label{sec:exi-fonc}

% Ce type d’exigences sera décrit en utilisant des diagrammes use case
% et en fournissant une description pour chacun d’entre eux (comme vu
% au cours théorique et aux exercices).

\subsection{Exigences non fonctionnelles}
\label{sec:exi-nonfonc}



\subsection{Exigences de domaine}
\label{sec:exi-dom}

\section{Besoins du systeme}
\label{sec:besoins-sys}

% Cette troisième section décrit en détail les fonctionnalités du
% système (à partir de celles décrites à la section précédente), sans
% aborder (dans la mesure du possible) *comment* elles doivent être
% réalisées.

\subsection{Exigences fonctionnelles}
\label{sec:exi-fonc-sys}

% Ce type d’exigences sera décrit en utilisant des diagrammes use
% case et en fournissant une description *plus détaillée* pour
% chacun d’entre eux (comme vu au cours théorique et aux exercices).


\subsection{Exigences non fonctionnelles}
\label{sec:exi-nonfonc-sys}



\subsection{Design et fonctionnement du systeme}
\label{sec:design}

% Cette partie sera décrite à l’aide des différents diagrammes UML
% vus au cours théorique et aux exercices.

\begin{figure}[ht]
  \centering
  \includegraphics[width=1\textwidth]{uml_files/UseCaseDiagram.png}
  \caption{\label{fig:usecase} Use Case Diagram du jeu en question}
\end{figure}

\begin{figure}[ht]
  \centering
  \includegraphics[width=1\textwidth]{uml_files/ClassDiagram.png}
  \caption{\label{fig:class} Diagramme des classes}
\end{figure}


\appendix

\section{Index des termes utilisés}
\label{sec:index}

\end{document}
