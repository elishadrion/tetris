\documentclass[11pt]{article}
\usepackage[french]{babel}
\usepackage[utf8]{inputenc}
\usepackage[T1]{fontenc}
\usepackage{amsmath}
\usepackage{amsfonts}
\usepackage{amssymb}
\usepackage{hyperref}


%opening
\title{Projet d'année}
%\author{Rémy Detobel}

\begin{document}


\maketitle


\section{Question}
  \subsection{Général}
    \begin{enumerate}
    \item Discussion en partie
    \item Discussion: channel public, avec l'adversaire, que avec ses amis
    \item Pouvoir bloquer une personne (que conséquences)
    \item Spectateur à la partie (et que peut il voir)
    \item Indiquer qu'en fonction du niveau d'énergie il peut encore joué ou non
    \item classement peut être affiché par quelqu'un qui n'est pas (connecté)
    \item authentification du compte (mot de passe)
    \item Tri des cartes lorsqu'on les consultes (si oui, comment: alpha, niveau d'énergie, ... ?)
    \item mettre un nom à un deck (numéro par défaut ?)
    \item (fichier qui enregistre en général doivent ils être exportable)
    \item choisir qui on veut avoir en duel (et non passer par le choix automatique)
    \item consulter la collection/deck de carte que quelqu'un d'autre (amis, n'importe qui)
    \item comment on choisir l'adversaire ?  N'importe qui ou en fonction du classement, du niveau, ... pour que des joueurs du même niveau se battent
    \end{enumerate}
  
  \subsection{Concernant le jeu}
    \begin{enumerate}
    \item maximum de cartes que l'on peut avoir en main
    \item carte joué sont elles défaussées (pourraient être récupérée) ou retiré du jeu
    \item avoir des cartes autrement que avec une victoire ?
    \item en cas de victoire quel carte gagne-t-on ?  Comment est elle choisie ?
    \item 20 cartes du début tirée aux hasard ou paquet par défaut
    \item lorsque l'on a 20 cartes est-ce que on a un deck automatique
    \item niveau pour les joueurs (en fonction de point qu'il gagnerait avec ses matches) et bonus si il change de niveau
    \item peut-on perdre des cartes (dans quel contexte/situation) et si oui peut-on descendre en dessous de 20 cartes.
    \item nombre maximum de créature sur le plateau
    \item attaque possible dès que la carte est posée ?
    \item ordre de priorité sur les attaques (choisir qui on attaque, dépendant des effets ?)
    \end{enumerate}
    
  \subsection{En plus}
    \begin{enumerate}
    \item échange de carte avec ses amis ou autre
    \end{enumerate}


    
    


\end{document}
